\documentclass[11pt]{article}
\usepackage[margin=1in]{geometry}
\usepackage{qtree}
%\usepackage{tikz, tikz-qtree}
\usepackage{amsmath, amsthm, algorithm, algorithmic}
\renewcommand{\labelenumi}{\alph{enumi})}
\newcommand{\comment}[1]{}
\begin{document}
\title{EECS340 - Algorithms - HW\#4}
\date{\today}
\author{Mark Schultz - mxs802}
\maketitle
\vspace{2in}

\section*{6.1-7}
The nodes past n/2 are leaves because the children, found at 2n and 2n+1 would be past the end of the array. Therefore they must be leaves because they cannot have any children.
{\centering\Tree [.A [.B [.D H I ] 
[.E J K ] ]
[.C [.F L ][.G ] ] ]\\}
\begin{center}

\begin{tabular}{|c|c|c|c|c|c|c|c|c|c|c|c|}
\hline
A & B & C & D & E & F & G & H & I & J & K & L \\ \hline
1 & 2 & 3 & 4 & 5 & 6 & 7 & 8 & 9 & 10 & 11 & 12 \\ \hline
n & n & n & n & n & n & l & l & l & l & l & l \\
\hline
\end{tabular}
\end{center}
\section*{6.3-3}
Assume there are $\lceil n/2\rceil \text{ leaves with }h=0$ \\
$f(T,h)$ is the number of nodes with height h in heap T. \\
$n(T)$ is the number of nodes in heap T \\
Hypothesis: $f(T,h)\leq \lceil \frac{n(T)}{{2^{h+1}}}\rceil$ \\
$f(n(T),h-1)\leq \frac{n(T)}{2^h}$ therefore, if the height of a node in T is h then the height of a node in T$\prime$ is h-1. \\
\begin{align*}
f(n(T),h)&=f(n(T\prime),h-1) \\
&\leq \lceil\frac{n(T\prime)}{2^h}\rceil \\
n(T\prime)&=\lfloor\frac{n(T)}{2}\rfloor \\
&\leq \lceil\frac{\lfloor\frac{n(T)}{2}\rfloor}{2^h}\rceil \\
&\leq \lceil\frac{n(T)}{2^{h+1}}\rceil
\end{align*}
\section*{6.5-6}
\begin{algorithmic}[1]
\WHILE{$i>1\text{ and }A[\textsc{Parent}(i)]<key$}
\STATE $A[i]=A[\textsc{Parent}(i)]$
\STATE $i=\textsc{Parent}(i)$
\ENDWHILE
\STATE $A[i]=key$
\end{algorithmic}
\end{document}

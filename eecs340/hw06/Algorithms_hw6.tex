\documentclass[11pt]{article}
\usepackage[margin=1in]{geometry}
\usepackage{amsmath, amsthm, algorithm, algorithmic}
\renewcommand{\labelenumi}{\alph{enumi})}
\newcommand{\comment}[1]{}
\begin{document}
\title{EECS340 - Algorithms - HW\#6}
\date{\today}
\author{Mark Schultz - mxs802}
\maketitle
\vspace{2in}
\section*{Problem 1}
\begin{algorithmic}[1]
\STATE \textsc{Sparse-Transpose}(R,C,V,m,n,k)
\FOR {$i=0$ to $k$}
\STATE $R\prime [C[i]] \leftarrow R\prime [C[i]]+1$
\ENDFOR
\FOR {$i=1$ to $k$}
\STATE $R\prime [i] \leftarrow R\prime [i] + R\prime [i-1]$
\ENDFOR
\FOR {$r \leftarrow m-1$ to $0$}
\FOR {$j \leftarrow R\prime [r+1]-1$ downto $R[r]$}
\STATE $C\prime [R\prime [C[j]]] \leftarrow r$
\STATE $V\prime [R\prime [C[j]]] \leftarrow V[j]$
\STATE $R\prime [C[j]] \leftarrow R\prime [C[j]]-1$
\ENDFOR
\ENDFOR
\end{algorithmic}
\section*{Problem 2}
\begin{algorithmic}[1]
\STATE \textsc{Stable-Sort(A)}
\FOR {$i=1$ to $n=A.length$}
\STATE $A[i]=A[i]-\frac{A[i]}{n+1}$
\ENDFOR
\STATE \textsc{Mystery-Sort(A)}
\FOR {$i=1$ to $n=A.length$}
\STATE $A[i]=ceil(A[i])$
\ENDFOR
\end{algorithmic}

The idea is to subtract an amount from the duplicates such that they get sorted in order but do not become a different number then after Mystery-Sort they are passed through the ceiling function to return the original integers.
\end{document}
